\documentclass{beamer}
\renewcommand\thesection{\arabic{section}}
\newcommand{\myfont}{\rmfamily\normalsize\upshape\mdseries}
\newcommand{\degree}{^\circ}
\newcommand{\R}{\mathbb{R}}
\title{\sffamily Exercises for Midterm 2}  
\institute[UM-SJTU JI]{University of Michigan-Shanghai Jiao Tong University Joint Institute}
\author{Kulu}
\usepackage{graphicx}
\usepackage{picinpar}
\usepackage{indentfirst}
\usepackage{chemformula}
\usepackage{geometry}
\usepackage{subfigure}
\usepackage{appendix}
\usepackage{amsfonts,amsmath,amssymb}
\usepackage{enumerate}
\usepackage{float}
\usepackage{geometry}
\usepackage{latexsym}
\usepackage{listings}
\usepackage{multicol,multirow,multido}
\usepackage{tabularx}
\usepackage{ulem}
\usepackage{tikz}
\usepackage{xcolor}
\usepackage{cite}
\usepackage{setspace}
\usepackage{hyperref}
\usepackage{textpos}
\usepackage{booktabs}

\usetheme[dove]{Boadilla}
\usecolortheme{dolphin}
\useoutertheme{miniframes}
\begin{document}
\usebackgroundtemplate{\tikz\node[opacity=0.05]{
        \centerline{\includegraphics[
                height=\paperheight]{kulu.jpg}}
    };}
\begin{titlepage}
    \begin{center}
        VV186 - Honors Mathmatics II
    \end{center}
\end{titlepage}
\myfont

\begin{frame}
    \frametitle{Frequently Asked Question in Sample Exam}
    \begin{figure}
        \centering
        \includegraphics[width=1\textwidth]{question.png}
    \end{figure}

\end{frame}

\begin{frame}
    \frametitle{Exercises}
    \hspace{1em}
    1. Suppose that $(V,+,\cdot)$ is a vector space and let $U,W\subset V$ be two
    subspaces. Then:
    \vspace{1em}
    \begin{center}
        \begin{itemize}
            \item[(A)]$U\cap W \neq \emptyset$
            \item[(B)]$V~ \backslash U$ is also a subspace of $V$
            \item[(C)]$U\cup W$ is a subspace of $V$
            \item[(D)]$U\cap W$ is a subspace of $V$
        \end{itemize}
    \end{center}
\end{frame}
\begin{frame}
    \frametitle{Exercises}

    \hspace{1em}
    2. Let $(a,b)\subset \R$ be an open interval and denote  $C^1(a,b)$ the vector
    space of continuously differentiable functions on $(a,b)$. On this space a norm is defined by
    \vspace{1em}
    \begin{itemize}
        \item[(A)]$||f||:=\underset{x\in (a,b)}{\sup} |f~(x)|$
        \item[(B)]$||f||:=\underset{x\in (a,b)}{\sup} |f~^\prime(x)|$
        \item[(C)]$||f||:=\underset{x\in (a,b)}{\sup} |f~(x)|+\underset{x\in (a,b)}{\sup} |f~'(x)|$
        \item[(D)]$||f||:=\underset{x\in (a,b)}{\sup} (|f~(x)|+|f~'(x)|)$
    \end{itemize}
\end{frame}
\begin{frame}
    \frametitle{Exercises}
    \hspace{1em}
    3. Let $(a_n)$ be a sequence of real numbers such that
    $f~(x)=\sum_{n=0}^\infty a_n x^n$ converges at least for
    $x \in [0,1]$. Then $f~(x)$
    \vspace{1em}
    \begin{itemize}
        \item[(A)] $f~(x)$ must converge for $x=-1/2$
        \item[(B)] $f~(x)$ must converge for $x=-1$
        \item[(C)] $f~(x)$ may or may not converge for $x=-1$
        \item[(D)] $f~(x)$ never converges for $x=2$
    \end{itemize}
\end{frame}

\begin{frame}
    \frametitle{Exercises}
    \hspace{1em}
    4.
    $$ f(x)=\left\{
        \begin{aligned}
            x^4\sin \frac{1}{x} \quad x\neq 0 \\
            0  \quad x=0
        \end{aligned}
        \right.
    $$
    Calculate $f~^{\prime\prime}(x)$
\end{frame}


\begin{frame}
    \frametitle{Exercises}
    \hspace{1em}
    5. Suppose that $f: \R \to \R$ is twice differentiable and
    has a local minimum at $x=0$. Then
    \vspace{1em}
    \begin{itemize}
        \item[(A)] $f$ is convex in a neighborhood of $x=0$
        \item[(B)] $f~''(0)>0$
        \item[(C)] $f~''(0) \geq 0$ and $f~''(0)=0$ is possible
        \item[(D)] $f~'(x)$  is increasing in a neighborhood of $x=0$
    \end{itemize}
\end{frame}
\begin{frame}
    \frametitle{Exercises}
    \hspace{1em}
    6. Let $V$ be a vector space over $\mathbb{F}=\R \text{ or } \mathbb{C}$.
    If one were to define
    $$U_1+U_2:= \{ z\in V : \underset{x\in U_1}{\exists} \underset{y \in U_2}{\exists} z=x+y \}$$
    $$U_1-U_2:= \{ z\in V : \underset{x\in U_1}{\exists} \underset{y \in U_2}{\exists} z=x-y \}$$
    for subspaces $U_1$, $U_2$ of $V$, then one would have:
    \vspace{1em}
    \begin{itemize}
        \item[(A)] $U_1-U_2 \subset U_1$
        \item[(B)] $U_1\subset U_1+U_2$
        \item[(C)] $(U_1-U_2)+U_2=U_1$
        \item[(D)] $U_1-U_2=U_1+U_2$
    \end{itemize}
\end{frame}
\begin{frame}
    \frametitle{Exercises}
    \hspace{1em}
    7. (i) Does the following series converge?
    $$\sum_{n=0}^\infty \frac{(2n)!(3n)!}{n!(4n)!}$$
    \hspace{2em}
    (ii) For which $a \in \R$ does the following series converge?
    $$\sum_{n=0}^\infty (\frac{1}{n}-\sin(\frac{1}{n}))^a$$
\end{frame}
\begin{frame}
    \frametitle{Exercises \textcolor{red}{Very likely to appear in mid2}}
    \hspace{1em} 8. Let $f: [0,2\pi ] \to \R$ be given by
    $$f~(x)=\frac{1}{1+e^{\pi-x }\sin (x)}$$
    \vspace{1em}
    \begin{itemize}
        \item[(i)] For which $x$ is $f~'(x)=0$? Derive the solution to the equation $\sin(x)=\cos(x)$, $x \in [0,2\pi]$
        \item[(ii)] Where is $f $ increasing? Where is $f $ decreasing?
        \item[(iii)] Find the \textcolor{red}{local extrema} of $f$
        \item[(iv)] What can you say about the convexity and concavity of $f$?
        \item[(v)] Sketch the graph of $f$, clearly indicating any siginicant features of the graph
    \end{itemize}
\end{frame}
\begin{frame}
    \frametitle{Exercises}
    \hspace{1em}
    9$^*$. Suppose that $(f_n)$ is a sequence of increasing functions
    $f_n: [0,1] \to [0,1]$, $n \in \mathbb{N}$, such that
    $$\underset{n\to \infty}{\lim} f_n(x) =f~(x) \text{ ~~~~~~~~~~~~~~~~ for all } x \in [0,1],$$
    Suppose that $f$ is a continuous function. Show that the convergence is uniform.
    \par \textbf{(Note that the functions $f_n$ are not assumed to be continuous.)}
    \\ \vspace{2em}
    Comment. This is also called \itshape Dini's theorem.\myfont
\end{frame}
\begin{frame}
    \frametitle{Exercises}
    \hspace{1em}
    10. Let $(f_n)$ be a sequence of functions in $C([a,b])$,
    and $(f_n)$ converges to some function $f$ uniformly.
    Prove that if $f\neq 0$ on $[a,b]$, then $(\frac{1}{f_n} )$ converges to $\frac{1}{f}$ uniformly.
\end{frame}

\begin{frame}
    \frametitle{Exercises \textcolor{red}{Very likely to appear in mid2}}
    \hspace{1em}
    11. From Sample Test, but professor didn't give answers for this exercise.
    \begin{figure}
        \centering
        \includegraphics[width=1\textwidth]{exercise.png}
    \end{figure}
\end{frame}

\begin{frame}
    \frametitle{End}
    \centering
    \LARGE{Good Luck!}
\end{frame}

\begin{frame}
    \frametitle{Reference}
    \begin{itemize}
        \item Exercises from 2020 Vv186 Midterm 2.
    \end{itemize}
\end{frame}

\end{document}